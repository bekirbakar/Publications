\documentclass[conference, a4paper]{IEEEtran}

\IEEEoverridecommandlockouts

\usepackage[turkish]{babel}
\usepackage[utf8]{inputenc}
\usepackage[T1]{fontenc}
\usepackage{cite}
\usepackage[ruled,vlined]{algorithm2e}
\usepackage{booktabs}
\usepackage{amsmath}
\usepackage{url}
\usepackage{multirow}
\usepackage{array}
\usepackage{xcolor}
\usepackage[tight,footnotesize]{subfigure}
\usepackage[lofdepth,lotdepth]{subfig}
\usepackage{multicol}
\usepackage{subfigure}

\SetAlgorithmName{Algoritma}{}{}

\DeclareMathOperator*{\argmax}{arg\,max}

\ifCLASSINFOpdf
  \usepackage[pdftex]{graphicx}
\else
\fi

\usepackage{svg}

\hyphenation{op-tical net-works semi-conduc-tor}

\setlength{\textfloatsep}{5pt}

\AtBeginDocument{\renewcommand\tablename{TABLO}}

\AtBeginDocument{\renewcommand\abstractname{Abstract}}

\SetKwInput{KwInput}{Girdi}                
\SetKwInput{KwOutput}{Çıktı}   
\SetKwIF{If}{ElseIf}{Else}{If}{}{else if}{}{end if}

\begin{document}

\IEEEpubid{\makebox[\columnwidth]{978-1-7281-7206-4/20/\$31.00 ©2020 IEEE\hfill}
    \hspace{\columnsep}\makebox[\columnwidth]{}}

\title{Türkçe Kural Tabanlı Resmi Doküman Tipi Tespiti\\Turkish Rule-Based Official Document Type Detection}

\author{\IEEEauthorblockN{Bekir BAKAR\IEEEauthorrefmark{1},
        Filiz AKSOY\IEEEauthorrefmark{2}, Apdullah YAYIK\IEEEauthorrefmark{3},\\
        Sevcan İÇÖZ\IEEEauthorrefmark{4} ve Vedat AYBAR\IEEEauthorrefmark{5}}
    \IEEEauthorblockA{Mobildev\\
        İstanbul, Türkiye\\
        \{bekir.bakar\IEEEauthorrefmark{1},
        filiz.aksoy\IEEEauthorrefmark{2},
        apdullah.yayik\IEEEauthorrefmark{3},
        sevcan.icoz\IEEEauthorrefmark{4},
        vedat.aybar\IEEEauthorrefmark{5}\}@mobildev.com}

    \and
    \IEEEauthorblockN{Tunga GÜNGÖR}
    \IEEEauthorblockA{Bilgisayar  Mühendisliği Bölümü
        \\Boğaziçi Üniversitesi
        \\İstanbul, Türkiye
        \\gungort@boun.edu.tr
    }
}

\maketitle

\begin{ozet}
    Bu çalışma, kurumların 2016 yılında Türkiye’de yürürlüğe giren 6698 sayılı Kişisel Verilerin Korunması Kanununa
    (KVKK) uyum sağlamalarına yardımcı olmak amacıyla, resmi dokümanlardaki kişisel bilgileri ve aralarındaki
    ilişkileri tespit etmek için geliştirilen DATAMIN isimli ürünün ilk aşamasını kapsamaktadır. Dokümanlarda bulunan
    alan isimlerinin ayırt edici etki değerleri belirlenerek, esnek olarak tasarlanmış düzenli ifadeler ve en az
    düzeltme mesafesi ile eşleşme kontrolüne dayalı, kural tabanlı resmi doküman tipi tespitine yönelik metot
    geliştirilmiştir. Önerilen metodun kaliteli optik karakter tanıma işleminin mümkün olduğu durumlarda son derece
    etkili olduğu ve doğru modelleme yapabildiği tespit edilmiştir.
\end{ozet}

\begin{IEEEanahtar}
    KVKK, DATAMIN, doküman tipi tespiti, düzenli ifadeler, en az düzeltme mesafesi.
\end{IEEEanahtar}

\begin{abstract}
    This study is the first stage of industrial application that will be used in the product named DATAMIN, which is
    being developed to help companies adapt Personal Data Protection Law (DPL) No. 6698 came into force in 2016 in
    Turkey, by extracting and relationing personal information in official documents. Rule-based official document type
    detection method based on matching control with flexible regular expressions and minimum edit distance was
    developed by determining the distinctive effect values of the field names in the documents. It was found that
    proposed method was highly effective and able to make accurate modeling when optical character recognition with
    high-quality was avaliable.
\end{abstract}

\begin{IEEEkeywords}
    DPL, DATAMIN, document type detection, regular expressions, minimum edit distance.
\end{IEEEkeywords}

\IEEEpeerreviewmaketitle

\IEEEpubidadjcol

\section{G{\footnotesize İ}r{\footnotesize İ}ş}
\label{sect:giris}
Kurum ve kuruluşlar çeşitli işlemleri yerine getirebilmek için, temas kurdukları şahıslara ait kişisel verileri kayıt
altına almakta ve işlemektedir. Bu süreçler, başta özel hayatın gizliliği olmak üzere kişilerin temel hak ve
özgürlüklerini korumak için kişisel verileri işleyen gerçek ve tüzel kişilerin yükümlülüklerini belirlemek amacıyla
Avrupa Birliği ülkelerinde Genel Veri Koruma Kanunu $-$General Data Protection Regulation, GDPR
\cite{otto2018regulation}, Türkiye Cumhuriyetinde ise 6698 sayılı Kişisel Verilerin Korunması Kanunu (KVKK) ile
\cite{turkiye2016title} düzenlenmiştir. Bahsi geçen kanunların getirdiği hukuki ve ekonomik yaptırımlar, kurum ve
kuruluşları kişisel bilgileri düzenleme ve kontrol altına almak için çözüm arayışına yöneltmiştir. Bu çalışmanın amacı,
doğal dil işleme yöntemleri ile KVKK kapsamında kişisel veri barındıran resmi dokümanların sınıflandırılmasıdır.
Kişisel veri içeren doküman tipinin belirlenmesi için kural tabanlı olarak geliştirilen sistem, her bir doküman tipine
özgün öznitelikler esas alınarak tasarlanmış düzenli ifadeler ve en az düzeltme mesafesi ile içerik tabanlı eşleşme
kontrolüne dayanmaktadır. İçerik tabanlı sınıflandırılması amaçlanan dokümanların çoğunluğu e-devlet üzerinden temin
edilebilen ve kişisel bilgi içeren 11 adet doküman tipidir. Bu dokümanlar aşağıda belirtilmiştir.

\begin{itemize}
    \setlength{\itemsep}{0.5pt}
    \item\label{idendity}Eski / Yeni Kimlik Belgesi (D-1 / D-2)
    \item\label{driver_license} Eski / Yeni Sürücü Belgesi (D-3 / D-4)
    \item Adli Sicil Belgesi (D-5)
    \item Öğrenci Belgesi (D-6)
    \item Nüfus Kayıt Belgesi (D-7)
    \item Araç Ruhsatı (D-8)
    \item Mezun Belgesi (D-9)
    \item Yerleşim Yeri ve Adres Bilgileri Belgesi (D-10)
    \item Askerlik Durum Belgesi (D-11)
\end{itemize}
burada D, doküman numarasını temsil etmektedir.

Bölüm \ref{sec:gecmis}'de mevcut yöntemler ve endüstriyel çözümler ele alınmıştır. Bölüm \ref{sec:materyal}'de
kullanılan metotlar ve karar verme süreci, Bölüm \ref{sec:result}'de ise deneysel sonuçlar ve değerlendirmeler
bulunmaktadır.

\section{Geçm{\footnotesize İ}ş Çalışmalar}
\label{sec:gecmis}
Geçmiş yıllarda birçok doküman sınıflandırma yöntemi önerilmiştir. Bunlardan en çok karşılaşılan resim türündeki
dokümanların yapısal yerleşim benzerliğini kullanan çalışmalardır. Bu yöntem kullanılarak sayısal resim makina
öğrenmesi tabanlı yaklaşımlar ile sınıflandırılmaktadır \cite{trstenjak2014knn, jones2004idf, csahin2017turkish}.
Ayrıca resim türündeki dokümanlar için optik karakter tanımaya (OKT) bağımlı olan içerik metni tabanlı benzerlik
kullanımına dayalı çalışmalar da mevcuttur. Bu yöntemde ise terim frekansı$-$ters doküman frekansı (TF$-$TDF) metodu
ile kural tabanlı \cite{han2000centroid} yaklaşımlar kullanılmaktadır. Ayrıca makina öğrenmesi yöntemlerinden k-en
yakın komşuluk \cite{trstenjak2014knn}, destek vektör makinaları \cite{csahin2017turkish}, na\"ive bayes
\cite{yoo2015classification} ile de sınıflandırmalar yapılmaktadır. Han ve arkadaşları \cite{han2000centroid}
tarafından önerilen metot, verilen bir dokümanın her bir kelimesinin TF$-$TDF değeri ile oluşturulan ve o dokümanı
temsil eden  doküman vektorünün daha önceden belirlenen doküman tipleri için oluşturulmuş olan doküman vektörleri ile
olan kosinüs benzerliği bilgisine dayalıdır. Bu kural tabanlı çalışmanın deneysel sonuçlarına bakıldığında makina
öğrenmesi yaklaşımlarından daha üstün olduğu görülmektedir. Reklam, mektup, özgeçmiş gibi 16 adet sınıfa ait toplam
$400.000$ adet resim içeren RVFL-CDIP \cite{harley2015evaluation} veri seti ile 2017 yılında Chris ve arkadaşları
\cite{kang2014convolutional} evrişimli yapay sinir ağı modeli ile başarılı sınıflandırma gerçekleştirmişlerdir. Türkçe
kişisel bilgi içerikli resmi doküman sınıflandırma problemi için mevcut benzer bir veri setinin bulunmadığı, bu sebeple
benzer çalışmaların yapılamadığı gözlemlenmiştir. Endüstride karşılaşılan IBM firmasının Stored IQSuite, Amazon
firmasının Macie, Sipiron firmasının ve Titus firmasının Illuminate ürünlerinin, Türkçe resmi doküman sınıflandırmaya
çözüm getirmediği görülmüştür.

\section{Materyal ve Metotlar}
\label{sec:materyal}
Kamera veya tarayıcı aracılığıyla digital ortama aktarılan dokümanlarda genişleme, doğrusal normalizasyon ve Otsu
adaptif eşikleme \cite{otsu1979threshold} metotları ile iyileştirmeler yapılmıştır. Ardından
tesseract\footnote{https://github.com/tesseract-ocr/tesseract} OKT kütüphanesi ile bilgi çıkarımı yapılmıştır.
İlerleyen alt bölümlerde kullanılan metotlar, etki faktörü tanımlaması ve karar verme süreçleri anlatılmıştır.

\subsection{Düzenli İfadeler ve En Az Düzeltme Mesafesi}
Düzenli ifadeler esas olarak karakter dizisi tespiti için cebirsel gösterimlerden oluşmaktadır. Büyük veri ambarında
örüntüsü belirli olan kelime yapıları tespiti ve geri dönüşümü için kullanılmaktadır \cite{jurafsky2000speech}. En az
düzeltme mesafesi ise iki farklı kelimenin biribirlerine dönüştürülebilmesi için ihtiyaç duyulan en az düzeltme işlemi
(ekleme, silme ve değiştirme gibi) sayısı olarak tanımlanır \cite{levenshtein1966binary}.

\subsection{Etki Değeri Tanımlaması}
Her bir doküman için Şekil \ref{fig:alanisimleri}'de örnek olarak gösterilen alan isimleri öznitelik olarak
kullanılmıştır. Her alan isminin her doküman tipi için karar verme işlemini etkileyen farklı bir belirleyici etki
değeri vardır. Eğer "memleket" alan ismi sadece bir adet dokümanda mevcut ise etki değeri, birden fazla dokümanda
mevcut olan "soyadı" alan isminin etki değerinden yüksek olmalıdır. Ayrıca, bu değer toplam 10 adet alan ismi bulunan
mezuniyet belgesi için 8 adet alan ismi bulunan kimlik belgesinden daha düşük olmalıdır. Alan isimlerinin etki
değerinin ne seviyede olduğu ancak ilgili uzayının bir arada değerlendirilmesi ile belirlenebilir. Bu durumda aynı alan
isimlerinin farklı doküman türlerinde mevcut olabilme durumu (TDF) ve dokümanların toplam alan sayılarının göz önünde
bulundurulması gereklidir.

\begin{figure}
    \centering
    \shorthandoff{=}
    % \subfigure[]{\includesvg[scale=0.55]{Images/01.svg}} \shorthandon{=}}
    \subfigure[]{\includegraphics[width=5.5cm, height=3.6cm]{./Images/01.png}} % width=2, scale=0.39
    \par\medskip
    % \subfigure[]{\includesvg[scale=0.25]{./Images/02.svg} \shorthandon{=}}
    \subfigure[]{\includegraphics[width=3.2cm, height=1.9cm]{./Images/02.png}}
    \caption{Alan isimleri yeşil dörtgen ile gösterilen 2  adet örnek doküman; (a) mezuniyet belgesi, (b) TC kimlik
        kartı}
    \label{fig:alanisimleri}
\end{figure}

\begin{algorithm}
    \caption{Ters Frekans Değerleri Tespiti}
    \label{alg:tersfrekans}
    \SetAlgoLined
    \KwInput{$\mathbf{A} = [a_1, a_2, \dots, a_{n}], \mathbf{D} = [d_1, d_2, \dots, d_{m}]$}
    \KwOutput{$\mathbf{F} = [f_1, f_2, \dots, f_{n}]_{ilk\_atama=0}$}
    \For{each $a_i$}{
        \For{each $d_j$}{
            \eIf{$a_i$ in $d_{j}$}{
                $f_i \leftarrow f_i + 1$
            }{}
        }
    }
\end{algorithm}

Belirtilen gereklilikler ve kısıtlar değerlendirilerek, tüm uzayda bulunan eşsiz alan isimlerine ters frekans değeri
ataması Algoritma \ref{alg:tersfrekans}'de gösterildiği gibi yapılmıştır. Burada, $\mathbf{A}$ benzersiz alan
isimlerini, $\mathbf{D}$ doküman tiplerine ait alan isimlerini ve  $\mathbf{F}$ ise benzersiz alan isimlerine ait ters
frekans değerlerini temsil etmektedir. $\mathbf{D}$'nin her elemanı bir adet doküman tipine ait alan isimlerini içeren
dizidir. Örneğin Şekil \ref{fig:alanisimleri}-b'deki TC kimlik kartı doküman tipi $d_1$ dizisi ile temsil edilmektedir.
Bu dizinin ilk $3$ elemanı ise şöyledir: $d_{11}$ = TC Kimlik No/Identity No, $d_{12}$ = Soyadı/Surname, $d_{13}$ =
Ad/Given Name. Sadece Şekil \ref{fig:alanisimleri}'de bulunan $2$ adet doküman tipi birlikte değerlendirildiğinde ters
frekans değeri "TC Kimlik No" alan ismi için $2$, "Diploma No" ve "Geçerlilik Tarihi" alan isimleri için ise $1$
olmaktadır.

Algoritma \ref{alg:tersfrekans}'de belirlenen her alana ait $\mathbf{F}$ ters frekans değerleri kullanılarak Algoritma
\ref{alg:etkidegeri}'de herbir doküman tipinde bulunan alan isimlerinin etki değerleri ($\mathbf{E}$) hesaplanmıştır;

\begin{equation}
    e_i = \frac{uzunluk(\mathbf{D})}{\mathbf{F}(d_{ij}) \times uzunluk(\mathbf{d_i})}
\end{equation}{}

\noindent burada $uzunluk(\mathbf{D})$ doküman tipi sayısını, $\mathbf{F}[d_{ij}]$ ters frekans değerini ve
$uzunluk(d_i)$ değeri doküman tipinin alan ismi sayısını belirtmektedir. Klasik TF$-$TDF de bulunan logaritma
işleminin kullanılmamasının sebebi etki değerlerinin $0$ veya negatif olmasını engellemektir.
Şekil \ref{fig:alanisimleri}-b'deki TC kimlik kartı dokümanına ait tüm alan isimlerinin ağırlıkları $e_1$ dizisi ile
temsil edilmektedir. Sadece Şekil \ref{fig:alanisimleri}'deki doküman tipleri bir arada değerlendirildiğinde Algoritma
\ref{alg:tersfrekans} ve \ref{alg:etkidegeri} kullanılarak elde edilen örnek etki alanları Tablo \ref{tablo:etki}'de
gösterilmiştir.

\begin{algorithm}
    \SetAlgoLined
    \KwInput{$\mathbf{D} = [d_1, d_2, \dots, d_m], \mathbf{F} = [f_1 , f_2, \dots, f_n]$}
    \KwOutput{$\mathbf{E}$}
    \For{each $d_i$}{
        $e_{i} \leftarrow [e_1, e_2, \dots, e_{h=uzunluk(d_i)}]_{ilk\_atama=0}$\\
        \For{each $d_{ij}$}{
            $e_{ij} \leftarrow uzunluk(\mathbf{D}) / ( \mathbf{F}[d_{ij}] \times uzunluk(d_i) )$\\
        }
    }
    \caption{Belirleyici Etki Değeri Tespiti}
    \label{alg:etkidegeri}
\end{algorithm}


\begin{table}
    \small
    \centering
    \caption{Örnek Etki Değerleri}
    \begin{tabular}{@{}cclcc@{}}
        \toprule
        \multicolumn{1}{l}
        {\begin{tabular}[l]{@{}l@{}}\textbf{Doküman} \\ \textbf{Tipi}\end{tabular}}                   &
        \multicolumn{1}{l}{\begin{tabular}[c]{@{}l@{}}\textbf{Alan} \\ \textbf{Sayısı}\end{tabular}}  &
        \begin{tabular}[c]{@{}l@{}}\textbf{Alan} \\ \textbf{Adı}\end{tabular}                         &
        \multicolumn{1}{l}{\begin{tabular}[c]{@{}l@{}}\textbf{Ters} \\ \textbf{Frekans}\end{tabular}} &
        \multicolumn{1}{l}{\begin{tabular}[c]{@{}l@{}}\textbf{Etki} \\ \textbf{Değeri}\end{tabular}}
        \\ \midrule \multirow{2}{*}{D-2} &
        \multirow{2}{*}{8}                                                                            &
        {\scriptsize TC Kimlik No}                                                                    & 2 &
        $\frac{2}{2 \times 8} = 0.125$                                                                      \\ &
                                                                                                      &
        {\scriptsize Geçerlillik Tarihi}                                                              &
        1                                                                                             &
        $\frac{2}{1 \times 8} = 0.250$
        \\
        \multirow{2}{*}{D-10}                                                                         &
        \multirow{2}{*}{10}                                                                           &
        {\scriptsize TC Kimlik No}                                                                    & 2 &
        $\frac{2}{2 \times 10} = 0.100$                                                                     \\ &  &
        {\scriptsize Diploma No}                                                                      & 1 &
        $\frac{2}{1 \times 10} = 0.200$                                                                     \\
        \cmidrule(l){1-5}
    \end{tabular}
    \label{tablo:etki}
\end{table}

\subsection{Benzeşme Oranı Tespiti ve Karar Verme İşlemi}
Benzeşme oranı belirlenmesinde, alan isimlerinden oluşturulan düzenli ifadelerin en az düzeltme mesafesi kriterlerine
göre eşleşmesi ve eşleşmemesi durumu göz önünde bulundurulmuştur. Alan isimleri için düzenli ifadeler, OKT esnasında
meydana gelebilecek Türkçe karakter bozuklukları göz önünde bulundurularak esnek olarak tasarlanmıştır. Örneğin Şekil
\ref{fig:alanisimleri}-b'de kimlik kartında "Önceki Soyadı" alan ismi bu işlem sonrasına "[OÖ]nceki ?Soyad[ıi]" düzenli
ifade örüntüsü olmaktadır. Örüntülerin tüm karakterlerinde büyük ve küçük harf olabilmesi sağlanarak eşleşmelerin
kapsamı artırılmıştır. Ayrıca, oluşturulan her bir düzenli ifade örüntü öncesi ve sonrasında kelime sınırı (boşluk veya
bir sonraki satıra geçiş) kısıtı eklenerek istenmeyen eşleşmeler önlenmiştir. En az düzeltme mesefesinde maliyetler her
bir işlem (silme, ekleme ve değiştirme) için $1$ olarak belirlenmiştir.

En az düzeltme mesafesi karakter sayısı az olan alan isimleri için yüksek seviyede belirlenir ise istenmeyen birçok
eşleşmeyi hatalı olarak yapacağından sistemi yanıltacaktır. Örneğin "soyadı" alan ismi için $3$ olarak belirlenirse
"sanatı" kelimesi ile hatalı eşleşme meydana gelecektir ($3$ değiştirme). Ayrıca karakter sayısı çok fazla olan alan
isimleri için çok düşük seviyede belirlenir ise sistemin esnekliği azaltılmış olur ve  gerçek eşleşmeler istenmeden
ihmal edilmiş olur. Örneğin "Geçerlilik Tarihi" alan ismi için $1$ olarak belirlenirse "Geçerlilih Tariki" kelimeleri
ile hatalı eşleşmeme meydana gelir. Bu sebeple düzeltme mesafesi deneysel incelemeler neticesinde aşağıdaki gibi
belirlenmiştir,
\begin{equation}
    \label{equation:ED}
    \mathbf{ED}(a_i) = \begin{cases}
        0 & \text{$a_i$'nin karakter sayısı} \leq 5      \\
        1 & 5 < \text{$a_i$'nin karakter sayısı} \leq 15 \\
        2 & 15 < \text{$a_i$'nin karakter sayısı}
    \end{cases}
\end{equation}
\noindent burada, $\mathbf{ED}$ en az düzeltme mesafesini ifade etmektedir. Düzenli ifade eşleşmesi ile en az düzeltme
mesafesi aynı sonlu durum otomata içerisinde birlikte çalışarak oldukça esnek ama aşırılıktan uzak bir biçimde doğru
eşleşme geri dönüşümü yapılabilecek şekilde tasarlanmıştır. Doküman tipi, ilk olarak Algoritma \ref{alg:benzesmeorani}
ile benzeşme oranı tespiti ve sonrasında karar verme işleminden sonra belirlenmiş olmaktadır. Algoritma
\ref{alg:benzesmeorani}'de tipi öğrenilmek istenilen dokümanın belirlenen doküman tiplerine ne kadar benzeştiği
ölçülmektedir. Burada, $\mathbf{Dok}$ tipi belirlenmek istenilen bir dokümanı, $\mathbf{E}$ Algoritma
\ref{alg:etkidegeri}'de hesaplanan etki değerlerini, $\mathbf{ED}$ en az düzeltme mesafesini ve $\mathbf{B}$ tipi
belirlenmek istenilen dokümanın benzeşme oranlarını ve $\mathbf{T}$ ise her doküman tipi için tam benzeşme oranlarını
göstermektedir. $\mathbf{RE}$ fonksiyonu giriş olarak alan isimlerini almakta düzenli ifadeler oluşturmaktadır.
$\mathbf{ED}$ en az düzeltme mesafesi değeri ise Denklem (\ref{equation:ED})'de belirtildiği gibi atanmıştır.
$\mathbf{KONTROL}$ fonksiyonu ise giriş olarak düzenli ifade, doküman ve en az düzeltme mesafesini almakta ve eğer
eşleşme var ise $1$, yok ise $0$ değerini döndürmektedir. Eşleşme olması durumunda benzerlik oranı ($b_i$) ilgili alan
ismine atanan etki değeri ($e_i$) olmaktır, aksi durumda ise ilk atanan $0$ değeri olmaktadır.

Algoritma \ref{alg:kararverme}'de doküman tipi kararı için tasarlanan işlemler verilmiştir. Burada  $\mathbf{s}$ eşik
değerini, $\mathbf{k}$ ise dokümana atanan sınıfı belirtmektedir. Öncelikle dokümanın en çok benzediği doküman türü
belirlenmektedir ($max(\mathbf{B})$). Eğer bu benzeşme değeri bu doküman türüne ait tam benzeşme değerine oranla
belirli bir sınırı aşıyor ise sınıf ataması yapılmaktadır, aksi durumda harici bir doküman olduğu kararı verilmektedir.
Bu çalışmada $\mathbf{s}$ eşik değeri deneysel olarak $0.25$ olarak belirlenmiştir. Örneğin TC kimlik kartı için tam
benzerlik oranı değeri $50$ olduğu durumda, tipi belirlenmek istenilen bir dokümanın benzerlik oranlarının en yüksek
değeri $12.5$ veya üzerinde ise TC kimlik kartı ataması yapılabilmektedir. Bu şart doküman tipi belirlemede yüksek etki
değerli alan isimlerinde eşleşme olmasını zorunlu kılarak sistemin karar vermede doğrulunu sağlayan önemli bir
faktördür.

\begin{algorithm}
    \caption{Benzeşme Oranı Tespiti}
    \label{alg:benzesmeorani}
    \SetAlgoLined
    \KwInput{$\mathbf{Dok}, \mathbf{ED}, \mathbf{E} = [e_1, e_2, \dots, e_m], \mathbf{D} = [d_1, d_2, \dots, d_m]$}

    \KwOutput{$\mathbf{B}= [b_1, b_2, \dots, b_{n}]_{ilk\_atama=0}, \mathbf{T}= [t_1, t_2,
                \dots, t_{n}]_{ilk\_atama=0}$}
    \For{each $d_i$}{
        \For{each $d_{ij}$}{
            \If{$ \mathbf{KONTROL}( \mathbf{RE}(d_{ij}), \mathbf{Dok}, \mathbf{ED}) )$}
            {$b_i \leftarrow b_i + e_{ij}$}{}
            $t_i \leftarrow t_i + e_{ij}$\\
        }
    }
\end{algorithm}

\SetKwInput{KwInput}{Girdi}
\SetKwInput{KwOutput}{Çıktı}
\SetKwIF{If}{ElseIf}{Else}{If}{}{else if}{else}{end if}


\begin{algorithm}
    \caption{Karar Verme}
    \label{alg:kararverme}
    \SetAlgoLined
    \KwInput{$\mathbf{B}, \mathbf{T}, \mathbf{s}$}
    \KwOutput{$\mathbf{k}$}
    \eIf{max($\mathbf{B}) \geq \mathbf{s} \times \mathbf{T}({arg\,max}(\mathbf{B}))$}
    {$\mathbf{k} \leftarrow {arg\,max}(\textbf{B})$}
    {$\mathbf{k} \leftarrow \text{diğer}$}

\end{algorithm}

\section{Sonuçlar}
\label{sec:result}
Bu çalışmada, Türkiye Cumhuriyeti'nde kullanılan resmi dokümanların kural tabanlı sınıflandırılması amaçlanmıştır.
Literatürde benzer bir yaklaşımın yapılabilmesi için herhangi bir veri setinin mevcut olmadığı görülmüştür. Söz konusu
dokümanların kişiye özel olmasından dolayı geliştirme ve test esnasında temin edilmesi hukuksal anlamda bir takım
prosedürlerin yerine getirilmesini gerektirmiştir. Bu sebeple kısıtlı olsa da gerçek veriler ile önerilen metotların
testi sağlanmıştır. Hazırlanan veri setinde doğrudan dijital ortamda oluşturulan 108 adet doküman ve kamera veya
tarayıcı aracılığı ile dijital ortama aktarılan 417 adet doküman bulunmaktadır. Bu 2 katagorideki veriler ayrı ayrı
test edilerek doğrulama matrisleri Şekil \ref{fig:dogrulamamatrisi}'de sunulmuştur.

\begin{figure}[th]
    \centering
    \shorthandoff{=}
    \subfigure{\includegraphics[scale=0.24]{./Images/03.png}}
    \par\medskip
    \subfigure{\includegraphics[scale=0.24]{./Images/04.png}}
    \shorthandon{=}
    \caption{Doğrulama matrisi (a) doğrudan dijital ortamda oluşturulan dokümanlar, (b) kamera veya tarayıcı
        aracılığıyla dijital ortama aktarılan dokümalar için. Burada $0$ etiketi harici doküman tipini, diğerleri ise
        Bölüm \ref{sect:giris}'de D ile belirtilen doküman tipi numaralarını temsil etmektedir.}
    \label{fig:dogrulamamatrisi}
\end{figure}

D-5, D-6, D-7, D-9, D-10 ve D-11 tiplerine ait dokümanlara doğrudan dijital ortamda oluşturturulmuş formda
karşılaşılabildiği için Şekil \ref{fig:dogrulamamatrisi}-a'da 6 adet doküman tipi sınırlandırması yapılmıştır. Test
sonrasında doküman tiplerinin tamamının doğru tespit edildiği görülmektedir (genel başarım \%100). Şunu özellikle
belirtmek isteriz ki, bu sonuç veri çıkarımının sorunsuz olarak gerçekleştirilebildiği (OKT bağımlılığı olmadan)
doküman tiplerinde önerilen metodun doğru karar verebildiğini ve doğru modelleme yeteneğine sahip olduğunu
göstermektedir.

Doküman tipinin hatalı atanması yerine mevcut doküman tiplerinin haricinde "diğer" doküman tipi şeklinde atama
yapabilabilmesi beklenmektedir. Örneğin, verilen bir TC kimlik kartı için eğer adli sicil kaydı ataması yapılırsa
bütüncül sistemin doğru çalışması mümkün olamayacaktır. Bu kısıt ve gereklilikler ışığında Şekil
\ref{fig:dogrulamamatrisi}-b incelendiğinde öncelikle test kümesinde bulunan $58$ adet harici tipteki dokümanın hatalı
olarak herhangi bir sınıfa atanmadığı görülmektedir. Ayrıca tipi hatalı olarak tespit edilen dokümanların neredeyse
hepsine yalnızca harici doküman tipi ataması yapılmıştır. Vurgulamak isteriz ki, bu durum karar verme algoritmasında
çalışmanın geri kalan bölümlerinde harici dokümanlar için yanıltıcı bir etkinin mevcut olmadığını göstermektedir. Şekil
\ref{fig:dogrulamamatrisi}-b incelendiğinde genel başarım \%49.16'dır. Ancak, "harici" doküman sınıfına hatalı
atamalar göz ardı edilirse \%98.55 olduğu gözlemlenmektedir. Kamera veya tarayıcı aracılığıyla dijital ortama aktarılan
dokümanların çözünürlük ve boyut gibi etkenlerden dolayı doğru okunabilmesi ağırlıklı olarak OKT kalitesine bağımlıdır.
Bu sebeple, Şekil \ref{fig:dogrulamamatrisi}-b'de doküman tipinin belirlenmesinde hatalar olduğu gözlemlenmiştir.

Çalışmanın devamı olarak başarım artırımı için ekibimiz tarafından evrişimli sinir ağları kullanılarak makina
öğrenmesi tabanlı bir sistem halihazırda geliştirilmektedir ve nihai durumda mevcut çalışmada önerilen metot ile hibrit
olarak çalışması planlanmaktadır. Bu çalışma, geliştirilmekte olan DATAMIN isimli KVKK uyum sürecine yardımcı olması
hedeflenen endüstriyel bir ürünün ilk aşaması olan kural tabanlı resmi dokuman sınıflandırma yaklaşımını içermektedir.
Yapılan kural tabanlı doküman sınıflandırma sonrasında belirlenen doküman tipine uygun makina öğrenmesi ve sözlük
tabanlı hibrit metotlar kullanılarak veri çıkarımı ile ilişkilendirmesi planlanmaktadır.

\section*{B{\footnotesize İ}lg{\footnotesize İ}lend{\footnotesize İ}rme}
Bu çalışma 3190083 numaralı 1051-TÜBİTAK Sanayi Ar-Ge Projeleri Destekleme Programı tarafından desteklenmiştir.

\bibliography{bibliography}
\bibliographystyle{ieeetr}

\end{document}
